\section{Introduction} 
%In hardware we are limited to one layer neural network (Perceptron). Perceptron can only learn linearly separable functions.  




% showed that using carefully selected detailed microarchitectural features and a simple  single-layered perceptron can provide a readily implementable solution. 





The performance of modern processors is heavily dependent on prediction and speculative execution throughout the pipeline.  Speculative execution, however has recently shown itself to be a two edged sword with the disclosure in early 2018 of microarchitectural attacks such as Spectre and Meltdown~\cite{spectre, meltdown}. The total cost to secure each component individually against all potential microarchitectural attack variations incur increasingly high cost in real estate, performance and energy~\cite{canella2018systematic, schwarz2019zombieload}, giving us an unsatisfying choice between expensive software mitigations or turning off these performance features altogether.  Indeed, future research in performance optimization, critical to the health of the computer industry, is also at risk if every advance risks exposing new vulnerabilities.


 Several of the mitigation either impose high performance overhead
to the system and/or only mitigate a specific variant. Detection mechanisms can provide information to the system to trigger further
mitigations to avoid their overhead unless an attack is suspected. However, this presents a particularly difficult
problem for simulation-to-real-world transfer: besides 
the
targeted vulnerable components of hardware changing in each variations of these attacks, the
system must also handle {\em evasive attacks} which poses multiple challenges for designing a resilient detector, specially resiliency against {\em adversarial machine learning attacks}.  

There are three main categories of evasion techniques: (1) bandwidth reduction (2) polymorphic evasion and (3) adversarial machine learning attacks. 
Software detectors have high performance overhead due to sampling and are also prone to bandwidth evasion~\cite{Gaudiot2020, PerSpectron}. Bandwidth evasion is based on timing the completion of all attack atomic 
tasks to fit within the sampling interval. Recent studies~\cite{PerSpectron, cyclone2019} on detection of transient, MDS and cache attacks and malware~\cite{Malware2015, ensembleRaid2015,kazdagli-16,RHMD2017} addressed the software limitations by moving the detection of
attacks to hardware, allowing thousand-times higher sampling frequency (100ms vs 1$\mu$s) without incurring performance overhead. It would be very difficult for the attacker to implement an microarcitectural attack atomic phase requiring an interval under three microseconds. 

Detection in hardware does not only make evasion based on bandwidth reduction very hard, but it is also shown to make detection more robust against polymorphic evasions~\cite{PerSpectron, cyclone2019, RHMD2017}. Polymorphic attacks is when the attacker attempts to produce different binaries implementing the attack. Because in microarchitectural attacks, attacker can implement instructions that will never get it’s result committed to the registers but force the CPU to execute instructions and leak data, it is important for security solutions 
to be able to detect attacks in the speculative execution feature 
space~\cite{wampler-19, PerSpectron}. Detection to hardware also allows the predictor to efficiently use and monitor a large set of informative microarchitectural features that is not limited to the commit. Recent study~\cite{PerSpectron} shows that low-level  features exist that are invariant under transformation and evasions by typical strategies
used by malware to evade signature based 
detectors~\cite{PaulKocher,paulKocherSpectreAttacks} and are accessible from hardware. Due to the number and nature of features used in PerSpectron, adversarial machine learning attacks described in prior hardware malware detector is potentially prohibitively expensive against PerSpectron. However, previous studies have not conclusively proved that the design cannot be evaded by adversarial machine learning attacks. 

Neural network models used in current work feature deep multi-layered  networks ({\em e.g.} RNN)  are not easily amenable to hardware due to design and runtime complexity. But Perceptron learning has shown to be implementable in hardware for various  applications including branch prediction, prefetching, replacement policies, and CPUadaptation~\cite{intelISCA2019}. Recent microarchitectures from Oracle~\cite{SPARCT4}, AMD ({\em e.g.} Bobcat, Jaguar, Piledriver, Zen, etc.), and Samsung~\cite{Mongoose,M3} are documented as featuring perceptron-based branch predictors.


Hinton~\cite{Hinton1985shape} 
finds that using the entire space of possible features in the 
training set made the 
mapping from the features’ instantiation parameters to the
object’s instantiation parameters became linear allowing the use of a simpler architecture, which could efficiently 
handle more complex images.   Similarly PerSpectron~\cite{PerSpectron} found microarchitectural features that when included in training, it could map a non-linearly separable problem to linear, which was then 
separable by perceptron and is readily implementable in hardware. The result was competitive 
with a more complex deepNN that is not easily implementable in hardware.


The science of defenses for machine learning based  detectors in hardware are somewhat less well developed. Here we consider several defensive goals. 

We propose a perceptron-based models for detection of broad range of microarchitectural attacks in hardware that outperforms current  detectors against adversarial machine learning attacks. 

%In this work, we introduce  





%and the selection of features from a larger set of event


% the attacks

% We believe that PerSpectron offers an inexpensive,
% always-on solution that can provide wide coverage against
% microarchitectural attacks. We also believe that there are
% interesting possibilities when combining PerSpectron with
% other defenses, for example to raise the alarm to activate other
% defenses to avoid their overhead unless an attack is suspected.










%  Using machine learning for detection of microarchitectural attacks, e.g., transient attacks~\cite{canella2018systematic,spectre,meltdown,koruyeh2018spectre} and cache-based covert channels~\cite{GrussFlushFlush,GrussFlushFlush, FlushReload2014Yarom,PrimeProbe2015last}, is a
%  promising mitigation direction~\cite{PerSpectron,RHMD2017, gulmezoglu2019fortuneteller,cacheBasedDetection2016Chiappetta,CloudRadar2016, BlackHatFogh,payer2016hexpads,ICCAD2015Wang, Duppel2013Zhang, mushtaq:cel-01824512}. 
%  This presents a particularly difficult
% challenge for simulation-to-real-world transfer: besides 
% the
% targeted components of hardware changing in each variations of these attacks, the
% system must also handle {\em evasive attacks} which poses multiple challenges for designing a resilient detector, specially resiliency against {\em adversarial machine learning attacks}. 


%Surveys of cybersecurity professionals indicate that this is the main reason causing low confidence in AI-based defense systems~\cite{CarbonBlack}.