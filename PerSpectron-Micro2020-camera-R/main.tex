%
\documentclass[conference]{IEEEtran}
\IEEEoverridecommandlockouts
% The preceding line is only needed to identify funding in the first footnote. If that is unneeded, please comment it out.
%\usepackage{cite}
\usepackage{amsmath,amssymb,amsfonts}
\usepackage{algorithmic}
\usepackage{graphicx}
\usepackage{textcomp}
\usepackage{xcolor}
\def\BibTeX{{\rm B\kern-.05em{\sc i\kern-.025em b}\kern-.08em
    T\kern-.1667em\lower.7ex\hbox{E}\kern-.125emX}}
    
    
    %%%%%%%%%%%%%%%%%%%%%%%%%%%%%%%%%
\usepackage{mathptmx} % This is Times font
\usepackage{adjustbox}
\usepackage{paralist}
\usepackage{verbatim}
\usepackage{fancyhdr}
\usepackage[normalem]{ulem}
\usepackage[hyphens]{url}
\usepackage[sort,nocompress]{cite}
\usepackage[final]{microtype}
%\usepackage{flushend}
\usepackage{listings}
\usepackage{ntheorem,lipsum}
\usepackage{adjustbox}
\usepackage{amsmath}
\usepackage{amsmath}
\usepackage{longtable}
\theorembodyfont{\upshape}
\newtheorem{definition}{Definition}
\usepackage{algorithmic}
\usepackage{graphicx}
\usepackage{textcomp}
\usepackage{xcolor}
\usepackage{setspace}
\usepackage{soul,color}
\usepackage{makecell}
\usepackage{adjustbox}
\soulregister\cite7
\soulregister\ref7
\soulregister\pageref7
\usepackage{titlesec}

\usepackage{caption} 
\captionsetup[table]{skip=8pt}

\newcommand{\scheme}{\textsc{APL}}

\newcommand{\ignore}[1]{}
\newcommand{\nael}[1]{{\color{green} Nael: \emph{#1}}} 
\newcommand{\samira}[1]{{\color{blue} samira: \emph{#1}}} 
\newcommand{\elba}[1]{{\color{orange} Elba: \emph{#1}}} 
\newtheorem{observation}{\textbf{Observation}}
\newtheorem{remark}{\textbf{Remark}}


\newenvironment{hlbreakable}%
  {\colorbox{yellow}}%
  {}
\def\BibTeX{{\rm B\kern-.05em{\sc i\kern-.025em b}\kern-.08em
    T\kern-.1667em\lower.7ex\hbox{E}\kern-.125emX}}
%--------------------------------------------------

% %%%%%%%%%%%%%%%%%%%%%%%%%%%%%%%%%%%%
\makeatletter
\newcommand{\linebreakand}{%
  \end{@IEEEauthorhalign}
  \hfill\mbox{}\par
  \mbox{}\hfill\begin{@IEEEauthorhalign}
}
\makeatother
 \begin{document}

\title{Adversarial-based Perceptron Learning For Microarchitectural Attack Detection in Hardware \\

}

%\setlength{\parindent}{4em}
\setlength{\parskip}{1em}



\maketitle
%--------Perspectron sections----------------
\begin{abstract}
\textcolor{red}{DMT: Title: 1. I think it works grammatically better without the "An".  2. I think it may
bury the main point a little.  The big point is better attack detection.  So something like " evasion-resistant attack detector using adversarial...", maybe? [caveat -- I haven't read enough of
the paper yet to know if this is a substantiated claim]}
Microarchitectural attack detection in hardware is attracting
increased research interest in recent years.  It has shown many advantages over software detection, such as low performance overhead, access to a larger set of detailed microarchitectural features, and greater robustness to multiple evasion techniques. The use of machine learning in hardware enables detection of attacks prior to leakage, which is essential. 
However,  defenses for hardware detectors against adversarial machine learning attacks have not been investigated, resulting in low confidence in the viability of implementing ML-based detection systems in hardware against microarchitectural attacks.

In this paper, we propose a non-injection and non-perturbation base transformation framework that generates microarchitectural attack samples 
from class conditioned generative adversarial networks.
Prior research has shown that training a classification model on augmented adversarial examples almost always improves the classifier's accuracy. 
Most adversarial example generation strategies in malware are based on developing a dynamic system that extrapolates
appropriate perturbations from benign data and adds them back onto those data based on the reverse-engineered model. We show that augmenting training sets with examples generated by conventional approaches, does not improve the detector resiliency against adversarial machine learning attacks.  Such augmentation fails because attacks perturbed from existing malware generate attacks that too closely resemble their unperturbed version and are not proved to fool hardware detectors and obviously limit the adversarial diversity.
 We overcome this problem with {\scheme}, a novel perceptron learning technique based on generative adversarial network theory~\cite{goodfellow2014generative} in which a perceptron-based detector is set up to play an adversarial game against a Deep Neural Network.  
 
 We then introduce dynamic perceptron weight clipping during adversarial training which ensures that the model does not rely on specific
features or unit of pipeline and to reduce the transferability of
the final classification parameter. Our final design significantly improves accuracy over state-of-art hardware detector for microarchitectural attacks (PerSpectron) with no extra hardware and performance overhead. Since perceptron has already been implemented in modern processors, 
our final design is readily implementable in hardware. 

%Traditional  approaches can be roughly split into to two classes, i


\end{abstract}
\begin{IEEEkeywords}
perceptron learning, microarchitectural attack defenses, secure architectures
\end{IEEEkeywords}
\section{Introduction} \label{intro}
%In hardware we are limited to one layer neural network (Perceptron). Perceptron can only learn linearly separable functions.  
% showed that using carefully selected detailed microarchitectural features and a simple  single-layered perceptron can provide a readily implementable solution. 

The performance of modern processors is heavily dependent on prediction and speculative execution throughout the pipeline.  Speculative execution, however has recently shown itself to be a two edged sword with the disclosure in early 2018 of microarchitectural attacks such as Spectre and Meltdown~\cite{spectre, meltdown}. The total cost to secure each component individually against all potential microarchitectural attack variations incur increasingly high cost in real estate, performance and energy~\cite{canella2018systematic, schwarz2019zombieload}, giving us an unsatisfying choice between expensive software mitigations or turning off these performance features altogether.  

\textcolor{red}{DMT: This builds up the background slowly, but I prefer a first paragraph that leads
with the punch line.  It should tell the reader what the main point is (eg, "this paper presents...", then you can give the background
starting in para 2.}
 
The proposed mitigations either impose high performance overhead
to the system and/or only mitigate a specific attack variant
~\cite{taram_csf19, intelanalysis, kiriansky2018dawg, domnitser2012non,saileshwar2019cleanupspec,retpoline, wang2007new, yu2019speculative, yan2018invisispec,CEASER,amd, koruyeh2019speccfi, arm_css}
A viable Detection mechanism is necessary as it would allow the system to trigger 
mitigations only as necessary when an attack is detected, thus avoiding their associated overhead~\cite{gulmezoglu2019fortuneteller, PerSpectron, cyclone2019}.
Prior works have detected microarchitectural attacks through two methods: Tracking unique cyclic properties in hardware ~\cite{cyclone2019} which can detect after the leakage completed, or detecting specific microarchitectural footprints of attacks using machine learning in hardware~\cite{PerSpectron, RHMD2017} and in software~\cite{gulmezoglu2019fortuneteller, cacheBasedDetection2016Chiappetta,
CloudRadar2016,
BlackHatFogh,
payer2016hexpads,
ICCAD2015Wang,
Duppel2013Zhang, 
mushtaq:cel-01824512}. 

More importantly, using machine learning enables detection of the attack during the calibration and training phases prior to leakage occurring, Contrary to static methods such as cyclic inference detection.  
However, this presents a particularly difficult
problem of simulation-to-real-world transfer: %besides the targeted vulnerable components of hardware changing in each variations of these attacks,
The detection system must also be resilient against multiple evasion techniques which include: (1) bandwidth reduction (2) polymorphic evasion and (3) adversarial machine learning attacks.   

 
Software detectors have high performance overhead due to limitations of sampling hardware performance counters in software and are also prone to bandwidth evasion~\cite{Gaudiot2020, PerSpectron}. Bandwidth evasion is based on timing the completion of all attack atomic 
tasks to fit within the detectors sampling interval. Recent studies~\cite{PerSpectron, cyclone2019} on detection of transient, MDS, cache attacks and malware~\cite{Malware2015, ensembleRaid2015,kazdagli-16,RHMD2017} addressed the software limitations by moving the detection of
attacks to hardware, allowing thousand-times higher sampling frequency (100ms vs 1$\mu$s) without incurring additional performance overhead. It has been shown that it is almost impossible to for an attacker to time a microarcitectural attacks atomic phases to occur between a sampling interval of under three microseconds. %Samira Cite Persperceron and Guidot here. 

Detection in hardware not only makes bandwidth evasion more difficult, but it also makes detection more robust against polymorphic evasions~\cite{PerSpectron, cyclone2019, RHMD2017}. Polymorphic evasion occurs when the attacker attempts to produce different binaries implementing the attack. In microarchitectural attacks, an attacker can implement instructions that will never have it’s result committed to the registers but still force the CPU to execute instructions and leak data.  For this reason, it is imperative to to detect attacks in the speculative execution feature 
space~\cite{wampler-19, PerSpectron}. 



Detection in hardware also allows the predictor to efficiently use and monitor a large set of informative microarchitectural features that are not available in software. Recent study~\cite{PerSpectron} shows that low-level features exist that are invariant under transformation and evasion strategies typically employed by malware to evade signature based detectors~\cite{PaulKocher,paulKocherSpectreAttacks}. Such features are only accessible in hardware. 

The ability to use high number of features in hardware without incurring performance overhead renders adversarial machine learning attacks
described in prior hardware malware detectors potentially prohibitively expensive against hardware detectors,
however, few studies have conclusively looked into methods to increase robustness of  microarchitectural attack detectors in hardware against adversarial machine learning attacks. 


Designing such a system in hardware poses multiple challenges: First, the available data from handcrafted adversarial attacks, and the existing real microarchitectural attack data is not sufficient for training and evaluation of a robust and reliable detector: While at least 350,000 new malware instances are being created and detected every day, only 20 distinct attack variations of microarchitectural attacks have been reported as of 2019. 
Previous detection mechanisms for microarchitectural attacks are trained and evaluated only on small samples, e.g., proof of concept code of current Spectre attacks and limited hand crafted adversarial perturbation mimicking the typical strategies 
used by malware to evade signature based
detectors~\cite{PaulKocher,paulKocherSpectreAttacks}. With such a small sample size of reported attacks to work with it is not realistic to expect a viable set of training data for ML-based microarchitectural attack detectors to exist for a long time. In this work we attempt to solve this problem. 



Second, augmenting the training dataset with adversarial examples~\cite{szegedy2014going, Goodfellow2015ADVexample, moosavidezfooli2016deepfool} has been proven to almost always improve the accuracy of machine learning detectors in other domains such as image classification. However,  projecting the adversarial perturbation to the program is not trivial in microarchitectural attacks. There is no direct mapping from microarchitectural feature values to program instructions (software code). Thus, generating adversarial microarchitectural attacks {\em manually} is both expensive and cumbersome, often requiring days to develop a single adversarial example capable of fooling a robust detector {\em i.e.,} and leak data before being flagged as suspicious. 

Adversarial machine learning examples can be generated {\em automatically}  or {\em manually}. Generating adversarial examples for microarchitectural attacks {\em automatically} has its own challenges.  
Unlike image pixels that can be easily manipulated without changing the image's visibility in the human eye, manipulating microarchitectural attacks is restricted by its functionality. 
Prior works in developing evasive malware~\cite{RHMD2017}, dynamically injected instructions to the malware binary to change the feature vector in a controlled way based on the reverse-engineered classifier to attempt to move the malware across the classification boundary. We will show in this work that data augmentation with examples produced by such conventional methods does not improve the accuracy of the microarchitectural attack classifier. 

A microarchitectural attacks' functionality depends profoundly on the victim microarchitecture being set by the attacker to a known state and the relative timing of instructions in the pipeline's fine-grain structures. 
Insertion of instructions should not interfere with attacks atomic tasks or otherwise can easily disrupt the state generated by attack channels and disable the attack. Yet when the attacker injects instructions carefully, {\em e.i.,} before and after attacks atomic tasks to ensure that the attack's functionality is not affected, we will see that the new microarchitectural samples do not add extra information to the model. This is because hardware detectors generalize based on the attack's footprint, which is observed only during the atomic tasks. Hardware microarchitectural attack detectors have 1000 times higher sampling frequency than software malware detectors. Hence, injected instructions similar to developing evasive malware can not hide the robust hardware detectors' footprint. Therefore,  techniques from evasive malware development are not sufficient for developing adversarial microarchitectural attacks. 

Forth,  
typical data augmentation technique using program synthesizers has many limitations to be used as adversarial example. For one, small modifications yields examples that do not diverge far from the original program. As a result, the additional examples do not add much variety to help the algorithm learn to generalize. In the case of microarchitectural attack detection, we want to see leakage happens using different technique, not just permutations of the same underlying attack.
In the case of microarchitectural attack detection, we want different examples of the same underlying vulnerability. Enriching a dataset with synthetic examples such as those produced by Generative Adversarial Networks (GANs)~\cite{goodfellow2014generative}(described in section~\ref{overview}), has potential to further enrich the available data beyond traditional augmentation techniques. While the recent advances in deep generative models~\cite{goodfellow2014generative} have brought a rapid progress in image recognition~\cite{}, relatively no progress have been made in microarchitectural design models. We believe this is due in part to hardware constraints {\em e.g., area, power, latency and performance overhead}. GANs are usually made of Convolutions Neural Network (CNNs), CNNs are not easily amenable in hardware. In our knowledge, no method has been shown previously to train a hardware friendly ML-based model  That is precisely our first goal to investigate. 

 \samira {Add dynamic model transformation motive here }
 
 Thus, the following framework sets our motivations:
  We have a data set of known   microarchitectural attack performance counter samples collected in hardware. 
 We assume that the observations have been generated according to some unknown distribution. 
 We design an automatic model that tries to mimic observed attack and safe programs footprints. If we achieve this goal , we can use the generated samples to train our detector offline. Our goals are: (1) to generate new attack examples that appear to have been drawn from seen attacks and (2) to generate examples that are suitably different from the observed attacks. In other words, our model shouldn't simply reproduce things it has already seen. (3) A method to control the category of microarchitectural attacks generated {\em controlling the transmit and recovery channel} during training. (4) A dynamic training method to ensure that the model does not rely on specific features or unit of pipeline and to reduce the  transferability of the final classification parameter. 
 
 
 
Finally, the science of defenses for machine learning based  detection in hardware against evasive attacks are somewhat less well developed. This is the main reason for low confidence in ML-based detection systems in hardware against microarchitectural attacks. In this paper we contribute on several defensive goals:

\textcolor{red}{DMT:"the main reason for low confidence..." likely true but impossible to substantiate,
so I'd say a little differently.  Maybe just "a major reason" is enough.}

\begin{enumerate}
\item We present the first use of adversarial learning to improve the
design of hardware detectors for microarchitectural attacks. 
\vspace{2mm}

\item We show that data augmentation using common program synthesizer does not improve the accuracy of detection, but training on adversarial examples significantly improves the accuracy of hardware detection for microarhitectural attacks. 

\item We design the first asymmetric  adversarial perceptron training, in which a simple one layer neural network that is amenable in hardware, can be trained  by a more complex deep neural network to detect adversarial machine learning attacks in hardware.
%\vspace{2mm}


\item We use the adversarial perceptron and a novel adaptive weight adjustment technique to produce \scheme{} which (1) significantly improves detection accuracy for broad range of microarchitectural attacks against evasive and adversarial machine learning attacks, and (2) it can be built in hardware with minimal performance and area overhead. 

\textcolor{red}{DMT:The last one sounds like just a summary of the others.  To make it a distinct 
contribution, focus a bit more on the tool and then maybe its okay:  "We present \scheme{},..."}
\end{enumerate}
 
\section{Background and Motivation}\label{background}
%summary of background


% \subsection{Evasive Attacks}
% It is natural to expect that an attacker would attempt to change their attack to avoid 
% detection using evasive attacks. 

% Bandwidth reduction mimicry is a common evasion strategy to slow down the rate of the 
% attack to weaken the microarchitectural signal. 
% Bandwidth evasion fits an attack atomic phase within a sampling window. 
% Software-only detection methods are also prone to bandwidth evasion. Software detectors have sampling intervals around 100 milliseconds, while Spectre requires 65 microseconds. Hardware detectors have much shorter intervals, in the range of 3 microseconds. A hardware detector could sample a Spectre phase 20 times, making it difficult to evade~\cite{PerSpectron}. Thus, bandwidth evasion is a problem with software detectors since sampling frequency is limited due to performance overhead in software. While in hardware, sampling interval can be decreased under fastest attacks atomic task length, with minimal performance overhead. 




% Bandwidth evasion is based on timing the completion of all attack atomic 
% tasks to fit within the sampling interval. It is a problem with the 
% low sampling frequency of the software detector. Li and 
% Gaudiot~\cite{Gaudiot2020} demonstrated that performance counter-based 
% detectors can be evaded by changing the bandwidth of the attack so that 
% it runs inside the 100 ms sampling interval of the detector. They identified 
% three "Atomic Tasks" that, if interrupted, will disable the attack: 
% (1) Flushing cache lines - 10 $\mu$s
% (2) Mistraining branch predictor - 13 $\mu$s
% (3) Attempting to infer the secret byte that is loaded into cache - 38 $\mu$s. 
% The authors concluded that putting the attack to sleep after all three tasks were 
% completed is the optimum evasion strategy for an attack.  This means that the 
% attack runs for 61 $\mu$s before being put to sleep, which allows Spectre to run 
% inside the 100 ms sampling interval of the detector. Our sampling interval is 
% 3 $\mu$s, which gives 20 sampling intervals within the 61 $\mu$s run time it takes 
% to complete all three tasks, making PerSpectron resistant to this evasion strategy.  
% The authors acknowledged that future work should be done on a dedicated hardware 
% detector to reduce performance overhead.
% Therefore evasion is made more difficult by decreasing the sampling 
% interval to below the run time of essential tasks of the attack.

% Polymorphic evasion is when the attacker attempts to produce different binaries implementing the attack. Examples of typical strategy
% used are: moving the leak to a function that cannot be inline or check the bounds with an \texttt{AND} mask, rather than \texttt{<}.
% Recent study shows that PerSpectron is resilient against typical strategy
% used by malware to evade signature based 
% detectors~\cite{PaulKocher,paulKocherSpectreAttacks}. However, they have not attempted to conclusively prove that the design cannot be evaded by adversarial ML attacks.

\subsection{Microarchitectural Attacks}

\subsection{Adversarial Attacks Against Machine Learning}

Recent literature has considered two types of threat models: black-box and white-box attacks. Under the black-box attack model, the attacker does not have access to the classification model parameters; whereas in the white-box attack model, the attacker has complete access to the model architecture and parameters, including potential defense mechanisms.
%(Papernot et al., 2017; Tramer et al., 2017; ` Carlini & Wagner, 2017). 




As previously mentioned, black-box adversaries have no access to the classifier
or defense parameters. It is further assumed that they do not have access to a large training dataset
but can query the targeted DNN as a black-box, i.e., access labels produced by the classifier for
specific query images. The adversary trains a model, called substitute, which has a potentially
different architecture than the targeted classifier, using a very small dataset augmented by synthetic
data labeled by querying the classifier. Adversarial examples are then found by applying any
attack method on the substitute network. It was found that such examples designed to fool the substitute often end up being misclassified by the targeted classifier~\cite{szegedy2014going, papernot2017practical}. In other words, black-box attacks are  transferrable from one model to the other. 

White-box models assume that the attacker has complete knowledge of all the classifier parameters, i.e., network architecture and weights, as well as the details of any defense mechanism. Given an input x and its associated ground-truth label y, the attacker thus has access to the loss function J(x, y) used to train the network, and uses it to compute the adversarial perturbation $\delta$. Attacks can be targeted, in that they attempt to cause the perturbed attack to be misclassified as safe target class. 

Unlike image classification there is no direct mapping from microarchitectural features value to software code.   
Projecting the adversarial perturbation $\delta$ to the program transformation is not trivial for both white-box and black-box attack model. 
%For example the number of \textit{commit.NonSpecStalls} can be 
Thus, generating adversarial microarchitectural attack manually is  expensive and cumbersome, often requiring days for developing a single adversarial example that would fool a robust detector {\em i.e.,} to leak data before the detector flags it as suspicious. 

Often handcrafted adversarial attacks, and even real microarchitectural attack data are not sufficient for training and evaluation of a robust and reliable detector: While at least 350,000 new malware instances are being created and detected every day, microarchitectural attacks were reported only over 20 distinct attack variations. 
Previous detection mechanisms for microarchitectural attacks are trained and evaluated only on small samples, e.g., proof of concept code of current Spectre attacks and limited hand crafted adversarial perturbation similar to typical strategy 
used by malware to evade signature based 
detectors~\cite{PaulKocher,paulKocherSpectreAttacks}. No teams of human can realistically design enough adversarial training data for ML-based microarchitectural attack detectors.
This is causing low confidence in AI-based defense systems against microarchitectural attacks.

Typical data augmentation technique using program synthesizers has many limitations. For one, small modifications yields examples that do not diverge far from the original program. As a result, the additional examples do not add much variety to help the algorithm learn to generalize. In the case of microarchitectural attack detection, we want to see leakage happens using different technique, not just permutations of the same underlying attack.
In the case of microarchitectural attack detection, we want different examples of the same underlying vulnerability. Enriching a dataset with synthetic examples such as those produced by GANs, has potential to further enrich the available data beyond traditional augmentation techniques. That is precisely our first goal to investigate. 

 
 The following framework sets our motivations:
  We have a data set of known attack samples $X$. 
 We assume that the observations have been generated according to some unknown distribution. 
 A generative model tries to mimic observed attack probability. If we achieve this goal , we can sample from the generated probability to generate observations that appear to have been drawn from known attacks. Our goal are: (1) to generate new attack examples that appear to have been drawn from seen attacks and (2) to generate examples that are suitably different from the observation in $X$. In other words, our model shouldn't simply reproduce things it has already seen. 


%\subsection{Handcrafting Adversarial Attack Strategies}

% The first technique is injection in which a malicious content is injected into benign process in order to avoid detection. 

% The downside of this technique is that the malicious Dynamic-Link Library (DLL) file must be stored on disk, which exposes it to detection by regular security solutions.

% To execute a malicious Dynamic-Link Library (DLL) under another process malware writes the path of a malicious DLL into a remote process’ address space. Then, to invoke the DLL’s execution, the malware creates a remote thread from the targeted process. This technique implies that the malicious DLL is stored on a disk before injecting it into the remote process. 



% To avoid storing the DLL on disk, Reflective DLL injection technique manually map the DLL’s raw binary into virtual memory, as the Windows loader would do, but without calling the Windows API’s LoadLibrary that might be detected by tools monitoring the LoadLibrary calls. It will be enough to get the correct address of the injected export function that will fully load and map remaining components of the DLL inside the target process, e.g. ReflectiveLoader().



% examples are executing DLL under another process, reflective DLL injection, hollowing the content of a benign process to include maliciois payload, 


%DLL injection is one of the simplest and most common processes injection techniques. To execute a malicious Dynamic-Link Library (DLL) under another process malware writes the path of a malicious DLL into a remote process’ address space. Then, to invoke the DLL’s execution, the malware creates a remote thread from the targeted process. This technique implies that the malicious DLL is stored on a disk before injecting it into the remote process.Steps for DLL injection:
%:Steps for DLL injection

% Locate the target process by traversing the running processes and call OpenProcess for obtaining a handle to it.
% Allocate the space for injecting the path of the malicious DLL file to the target process with a call to VirtualAllocEx with the targeted process handle.
% Write the path of the DLL into the allocated space with WriteProcessMemory.
% Retrieve the address of LoadLibrary from kernel32.dll, that given the path to DLL, loads it into memory (does not execute it though).
% Call CreateRemoteThread passing it the address of LoadLibrary causing the injected DLL file’s path to be loaded into memory and executed.
% The downside of this technique is that the malicious DLL file must be stored on disk, which exposes it to detection by regular security solutions. Nevertheless, this technique is employed by malware developers and is widespread in the wild. For example, Poison Ivy, a popular and long-standing RAT, uses DLL injection. Poison Ivy has been involved in several APT campaigns recommending itself as a tool of choice by APT groups for espionage operations.
%https://www.deepinstinct.com/2019/09/15/malware-evasion-techniques-part-1-process-injection-and-manipulation/#:~:text=Process%20injection%20and%20manipulation%20is%20a%20prominent%20method,undetected%20and%20launch%20and%20execute%20additional%20successful%20attacks.
 

\subsection{Defenses}
A popular approach to defend against adversarial attack in computer vision is to augment the training dataset with adversarial examples~\cite{szegedy2014going, Goodfellow2015ADVexample, moosavidezfooli2016deepfool}. Adversarial examples are generated using one or more chosen attack models and added to the training
set. This often results in increased robustness when the attack model used to generate the augmented
training set is the same as that used by the attacker. It tends to make the model more robust to white-box attacks than to black-box attacks due to gradient masking~\cite{Papernot2016TowardsTS, tramer2020ensemble}.

%2.2.2 DEFENSIVE DISTILLATION
Defensive distillation~\cite{papernot2016distillation} trains the classifier in two rounds using a variant of the
distillation~\cite{hinton2015distilling} method. This has the desirable effect of learning a smoother network
and reducing the amplitude of gradients around input points, making it difficult for attackers to
generate adversarial examples~\cite{papernot2016distillation}. It was, however, shown that, while defensive
distillation is effective against white-box attacks, it fails to adequately protect against black-box
attacks transferred from other networks~\cite{Carlini2017}.

Defenses against the proliferation of malware include Hardware Malware Detectors (HMDs)~\cite{RHMD2017}. They stochastically switch between different detectors. These detectors can be shown to be provably more difficult to
reverse engineer based on resent results in probably approximately
correct (PAC) learnability theory. 
%RHMD studied limited number of high level features and techniques used for attacking malware detectors. 


\subsection{Perceptron}
Unlike image information that is simple pixels, microarhitectural features are much more complex.PerSpectron includes features that capture the relationship between multiple features. Therefore no hidden layer is needed~\footnote{An example of such feature is the effect of misses and stalls 
in the Fetch stage. The squashed cycles in each stage, all the ROB, IQ, and 
Register full events, undone maps in the Rename stage, and memory order 
violation in the IEW stage propagate back to the Fetch stage. The relationship 
between these events' Fetch is not a simple cumulative function in an out-of-order 
processor. However, features such as \textit{fetch.MiscStallCycle} capture the 
relationship.}. The weights associated to these features have the potential to be updated further, similar to hidden weights in RNNs.
No hidden layer was necessary---the 
mapping from the features' instantiation parameters to the object's 
instantiation parameters became linear. 

In addition, PerSpectron showed that because the formation about attacks moves around the
processor, mutually correlated features of all components of processors should be included in the feature set in order to detect new variations of attacks. 
% Neural network models used     in current work feature deep multi-layered  networks ({\em e.g.} RNN)  are not easily amenable to hardware due to design and runtime complexity. 
% Hinton~\cite{Hinton1985shape} 
% finds that using the entire space of possible instantiation parameters in the 
% training set allowed the use of a simpler architecture, which could efficiently 
% handle more complex images.  Similarly PerSpectron showed that using carefully selected detailed microarchitectural features and a simple  single-layered perceptron can provide a readily implementable solution~\cite{PerSpectron}. 
% Perceptron learning has shown to be implementable in hardware for various  applications including branch prediction, prefetching, replacement policies, and CPUadaptation~\cite{intelISCA2019}. Recent microarchitectures from Oracle~\cite{SPARCT4}, AMD ({\em e.g.} Bobcat, Jaguar, Piledriver, Zen, etc.), and Samsung~\cite{Mongoose,M3} are documented as featuring perceptron-based branch predictors.

% Unlike image information that is simple pixels, microarhitectural features are much more complex.PerSpectron includes features that capture the relationship between multiple features. That's why hidden layer is not needed. ~\footnote{An example of such feature is the effect of misses and stalls 
% in the Fetch stage. The squashed cycles in each stage, all the ROB, IQ, and 
% Register full events, undone maps in the Rename stage, and memory order 
% violation in the IEW stage propagate back to the Fetch stage. The relationship 
% between these events' Fetch is not a simple cumulative function in an out-of-order 
% processor. However, features such as \textit{fetch.MiscStallCycle} capture the 
% relationship.} The weights associated to these features have the potential to be updated further, similar to hidden weights in RNNs.
% No hidden layer was necessary---the 
% mapping from the features' instantiation parameters to the object's 
% instantiation parameters became linear. In addition, PerSpectron showed that as the formation about attacks moves around the
% processor, mutually correlated features of all components of processors should be included in the feature set in order to detect new variations of attacks. 




 \subsection{Game setup}
\section{Threat Model}

% We assume an adversarial attack model which starts with the adversary attempting to reverse engineer the classifier. We assume that
% the attacker has access to a machine with a similar detector as the
% victim machine. This allows the attacker to observe the behavior
% of the classifier for given programs (whether malware or normal
% programs). With a model of the detector, the attacker can attempt to
% generate evading malware that hide themselves by changing some
% of their characteristics (feature values). 

% Such evading mechanism
%  is known as mimicry attacks~\cite{Mimicry2006,Mimicry2007}, which can be in the form of
%  no-op insertion, code obfuscation by the attackers, or calling benign
%  functions in the middle of the malicious payload~\cite{SCRAP2013HPCA}.
% We assume that the attacker that undertakes malware rewriting
% as part of a mimicry attack is interested in maintaining reasonable
% performance of the malware. If this assumption is not true, an attacker can simply run a normal program with embedded malware,
% that advances the malware program arbitrarily slow (e.g., 1 malware
% instruction every N normal instructions where N is arbitrarily large) making detection impossible. Note that this is a limitation of all
% anomaly detectors, and not only HMDs. This assumption is also
% reasonable for important segments of malware such as: malware
% that is time sensitive (e.g., that performs covert or side-channel attacks [16, 23, 37, 42]) and malware that is computationally intensive
% such as that executing on botnets being monetized under a pay-perinstall model [9] (e.g., Spam bots or Click fraud). Such malware have
% a utility to the malware writer proportional to their performance.


\section{ Overview}\label{alg}
\subsection{General Approach}
 
 GANs are capable of producing examples ranging from simple handwritten digits to photo-realistic images of human faces. However, although we could control the domain of examples our GAN learned to emulate by our selection of the training dataset. we could not specify any of the characteristics of the data samples the gan would generate. We could not control whether it would produce, say a new sample of adversarial meltdown attack. In image recognition, this concern may seem trivial.Because if the goal is to generate number 9, you can just keep generating until you get the number you want. However, for a domain of microarchitectural attacks the domain of possible answers gets too large for such a brute-force solution to be practical. In simple GAN, you have no control on what category of sample input will get produced. There is no way to direct the Generator to synthesize say an attack or safe program sample. Let alone other features such as the type of covert channel such as Flush+Reload or Flush+Flush.
 
 Our design has the ability to decide what kind of adversarial program will be generated. We could enter the descriptive features of attacks atomic tasks' samples into the generator and have it output a range of samples matching the criteria. It can greatly expedite the process of adversarial attack generation. We are sure there are many other practical applications where the ability to generate new attack sample that matches the channel type of our choice would a game changer. The CGAN was one of the first GAN innovations that made data generation possible. 
 
 CGAN is a generative adversarial learning whose generative and discriminator are conditioned during training by using some additional information e.g., labels. 
 The generator learns to produce realistic examples for each label in the training dataset and the discriminator learns to distinguish fake example-label pair from real-example pairs.   
 in contrast to a design where the Disriminator learns to assign a correct label to each real example in addition to distinguashing real example from fake, the discriminator does not learn to identify which class is which.  It learns to accept real matching pair while rejecting pairs that are mistmatched and pairs in which there are fake. 
 
 Accordingly in order to fool the discriminator it is not enough for the CGAN generator to produce realistic looking data. The example it generates also need to match their labels. After the generator is fully trained, this then allows us to specify what attack type we want the CGAN to synthesize by passing it the desired label. The generator uses the noise vector and label to synthesize a fake example (x given that or conditioned on y). The goal of this fake example is to look in the yes of Discriminator as close as possible to a real example for the given label.  
 
 
 
 
 
 
 
 
 
 
 
 
 
 
 
 
 In more technical terms, the Generator's goal is to produce examples that capture the characteristics of the training dataset,  so that the samples it generates look indistinguishable from the training data. The generator can be thought of as a microarchitectural attack detection model in reverse. Microarchitectural attack detection model learns the patterns of suspicious and safe activity in a program to discern an attacks footprints. Instead of recognising the pattern, the Generator learns to create them essentially from scratch; indeed, the input into the Generator is often no more than a vector of random numbers. %this can go in introduction 
 
 
 The Generator learns through the feedback it receives from the discriminator's classification. The discriminator's goal is to determine whether a particular example is safe (coming from the training dataset) or suspicious (created by the Generator). 
Accordingly, each time the discriminator is fooled into classifying an adversarial attack as safe, the generator knows it did something well. Conversely each time the discriminator correctly rejects a generator-produced adversarial as attack, the generator receives the feedback that it needs to improve.
The discriminator continues to improve as well. Like any classifier, it learns from how far its predictions are from the true class (safe or suspicious). So, as the generator gets better at producing realistic adversarial sample, the discriminator gets better at telling adversarial samples from safe programs data, and both models continue to improve simultaneously. 

GANs are notoriously hard to both train and evaluate. As with any other cutting-edge field, opinions about what is the best approach are evolving and no study has been done on microarchitectural data. 

We want to use GANs to create a large dataset, but we need a large dataset to train the GAN in the first place. 
Our solution is as follows:
first we use standard data augmentation using program synthesising to create a larger dataset. Second, we used this dataset to train a GAN to create synthetic examples. Third, we use the augmented dataset from step 1 along with the GAN-produced synthetic examples from step 2 to train the PerSpectron. The GAN model we used is a variation of Deep Convolutions GAN. We had to adjust the dimensions of hidden layers and the dimensions of the output from the Generator and input into the Discriminator PerSpectron. Then we used the trained PerSpectron to classify unseen test set.  

Training dataset-- The dataset of all safe programs and real attacks that we want the generator to learn to emulate with near-perfect quality. This dataset serves as input($x$) to the generator network. 

Random noise vector-- The raw input ($z$) to the Generator network. This input is a vector of random numbers that the generator uses as a starting point for generating adversarial examples. 

Discriminator network-- The discriminator takes as input either the real examples ($x$) coming from the training set or an adversarial example $x^{\star}$ produced by the generator. For each example, the discriminator determines and outputs the probability of whether the example is adversarial.

Iterative training/tuning--For each of the discriminators predictions, we determine how good it is--much as we would for a regular classifier-- and use results to attractively tune the discriminator and the generator networks through backpropogation; the discriminator's weights and biases are updated to maximize its classification and accuracy ( maximizing the probability of correct prediction: ($x$) as attack and $x^{\star}$ is safe. 
The generator's weights and biases are updated to maximize the probability that the discriminator missclassifies $x^{\star}$ as safe program. 


We have a theoretical understanding of why the symmetric GAN training should converge to the Nash equilibrium. We noticed that our asymmetric GAN traning has a lot higher gradient and so trianing happens much more quickly at start. Although theoretically there is a chance that the training might not converge at all, we empirically show that with the right stopping criteria, the trained model performs better than symmetric GAN training. 
But our dreadful sacrifice leads to significant improvement in accuracy.


Accuracy on 3-fold cross validation improved by 7\%


We take an attack sample first and project it back to the latent space $z$.
We use the generator $G$ to generate a similar example to $x$, called ${x^{\star}}$ by $G(z)$. Then we use the classifier $C$ to classify the example $C(x^{\star})$, which generally already tends to misclassify way less than running the classification directly on $x$. 

Performance of the new classifier does drop on the training set, but accuracy improves on test set including adversarial attacks. Figure shows APL achieves significant improvement in classification accuracy as the number of training examples increases, especially synthetic examples produced by DCGAN. The dotted line depicts classification performance for classic data augmentation using program synthesizers. The performance improves as the quantity of new (augmented) training examples increases; however, the improvement plateaus around the accuracy of 80\%, beyond which additional examples fail to yield improvement. 

The dashed line shows the additional increase in accuracy achieved by augmenting the dataset using GAN-produced synthetic examples. Starting from the point beyond which additional classically augmented examples stopped improving accuracy, we added syntetic data generated by our DCGAN. The classification performance improved from 83\% to pver 90\%, demonstrating the usefulness of \scheme. 

Figure shows the loss value of generator vs training epoch for a range of features. Using a few high level features would result in discriminator losing the game to the generator faster. we define the new metric {\em Nash Equilibrium Speed} to be the loss of generative model over 1000 epochs. that is proportional to the vulnerability to adversarial machine learning attack and  inversely proportional  to the resiliency of a model to adversarial machine learning attacks. 

People have noted that adding labels to the data—that is, to break it up into categories, almost always improves the performance of GANs.

Generated data, showing high features in .. which is high in all the variations. 




% Our results show adversarial training does not perform
% as well when a different attack strategy {\em i.e.} new variations of Speculative attack is used by the attacker. 

\subsection{GAN training algorithm}

For each training iteration
\subsubsection{Train the discriminator:}
(a) take a a random real example x from the training dataset. (b) Get a new random noise vector z and using the generator network synthesize a fake example  $x^{\star}$. (c) Use the discriminator network to classify  $x$ and  $x^{\star}$. (d) Compute the classification error and backpropogate the total error to update it's trainable parameters, seeking to minimize it's classification error. 

\subsubsection{Train the generator}
For each training iteration
(a) Get a new random noise z and using the generator network, synthesize fake example $x^{\star}$. (b) Use discriminator network to classify $x^{\star}$. (c) Compute the classification error and backpropogate the error to update the generators trainable parameters, seeking to maximize the discriminator's error. 

GANs reaching Nash equilibrium when the following conditions are met generator produces fake examples that are not distinguishable from the real data in the training dataset. The discriminator can at best randomly. The discriminator can at best randomly guess whether a particular example is fake or real.  

A conditional GAN allows us to direct the generator to synthesize the adversarial example we want. Although we could control the domain of example our GAN learned to emulate by our selection of the training dataset, we could not specify any of the characteristics of the data samples it is going to generate. GAN could synthesize realistic looking handwritten digits but we can not control what digits it produces. 
\subsection{Hardware Design}
Regardless of the order
and the position of each feature, the k-sparse representation
are distinctive, and only contain 0/1 values
% {\small
% \begin{align*}
% \vspace{0.05in}
% \tiny{[f1 = ReadResp, 
%  f2 = commitNonSpecStalls, 
%  f3 = PendingQuiesceStallCycles, 
%  f4 = CleanEvict]}
% \end{align*}}
% {\small
% \begin{align*}
% suspicious: <0,1,0,0> \\
% suspicious: <1,0,1,0> \\
% suspicious: <0,0,0,1>  \\ 
% safe: <1,1,0,0> 
% \end{align*}
% }
 Computing the perceptron output. Multiplication is not necessary to compute the dot product. Simply add the weight when the input bit is 1 and subtract when bit is -1. Only the sign bit of the result is needed to make a prediction. Prediction happens in hardware and in parallel to the execution.
No additional overhead for hardware counters. Hundreds of counters are included (for debug and verification). Simpler than perceptron-based branch predictors used in mobile CPUs (see e.g. the Samsung Exynos paper in ISCA2020).

\section{Methodology}\label{method}
In all experiments and evaluations we used 3-fold cross validation with case separation at
the patient level and each fold contained a balanced number
of attacks, and safe programs.

% Since
% our dataset was too small for effective training, we incorporated classic augmentation for the training process. We employed the DCGAN architecture to train each  class separately, using the same 3-fold cross validation process and the
% same data partition. 

The GAN model we used is a variation of Deep Convolutions GAN. We had to adjust the dimensions of hidden layers and the dimensions of the output from the Generator and input into the Discriminator PerSpectron. Then we used the trained PerSpectron to classify unseen test set.  

Training dataset-- The dataset of all safe programs and real attacks that we want the generator to learn to emulate with near-perfect quality. This dataset serves as input($x$) to the generator network. 

Random noise vector-- The raw input ($z$) to the Generator network. This input is a vector of random numbers that the generator uses as a starting point for generating adversarial examples. 

Discriminator network-- The discriminator takes as input either the real examples ($x$) coming from the training set or an adversarial example $x^{\star}$ produced by the generator. For each example, the discriminator determines and outputs the probability of whether the example is adversarial.

Iterative training/tuning--For each of the discriminators predictions, we determine how good it is--much as we would for a regular classifier-- and use results to attractively tune the discriminator and the generator networks through backpropogation; the discriminator's weights and biases are updated to maximize its classification and accuracy ( maximizing the probability of correct prediction: ($x$) as attack and $x^{\star}$ is safe. 
The generator's weights and biases are updated to maximize the probability that the discriminator missclassifies $x^{\star}$ as safe program. 



\begin{table}[!htbp]
\small
\centering
\begin{tabular}{|c|}
\hline
\textbf{Architecture}  \\ \hline
X86 O3CPU 1 core Single Thread at 2.0GHz \\ \hline

\textbf{Core}  \\ \hline
Tournament branch predictor\\
16 RAS entries, 4096 BTB entries\\
LQEntries=32, SQEntries=32, ROBEntries=192\\
fetch/dispatch/issue/commit width=8\\
numPhysIntRegs=256,numPhysFloatRegs=256 \\ \hline

\textbf{L1 I-Cache}  \\ \hline
32KB, 64B line, 4-way \\ \hline

\textbf{L1 D-Cache}  \\ \hline
64KB, 64B line, 8-way \\ \hline

\textbf{Shared L2 cache}  \\ \hline
2MB bank, 64B line, 8-way,  \\
mshrs=20, tgtsPerMshr=12, writeBuffers=8  \\ 
tagLatency=20, dataLatency=20, responseLatency=20\\ \hline
\end{tabular}
\caption{Parameters of simulated architecture}
  \label{table:GEM5}
\end{table}
\section{Security Analysis}\label{sec:whac}\label{sec:varSpec}

\subsection{Developing Evasive Microarchitectural Attacks }

\subsection{Comparison With Ground Truth}
\section{Experimental Evaluation}\label{results}

% No attack was able to leak, but number of false positive increased. 



Accuracy on 3-fold cross validation improved by 7\%


We take an attack sample first and project it back to the latent space $z$.
We use the generator $G$ to generate a similar example to $x$, called ${x^{\star}}$ by $G(z)$. Then we use the classifier $C$ to classify the example $C(x^{\star})$, which generally already tends to misclassify way less than running the classification directly on $x$. 

Performance of the new classifier does drop on the training set, but accuracy improves on test set including adversarial attacks. Figure shows APL achieves significant improvement in classification accuracy as the number of training examples increases, especially synthetic examples produced by DCGAN. The dotted line depicts classification performance for classic data augmentation using program synthesizers. The performance improves as the quantity of new (augmented) training examples increases; however, the improvement plateaus around the accuracy of 80\%, beyond which additional examples fail to yield improvement. 

The dashed line shows the additional increase in accuracy achieved by augmenting the dataset using GAN-produced synthetic examples. Starting from the point beyond which additional classically augmented examples stopped improving accuracy, we added syntetic data generated by our GAN. The classification performance improved from 83\% to pver 90\%, demonstrating the usefulness of \scheme. 

Figure shows the loss value of generator vs training epoch for a range of features. Using a few high level features would result in discriminator losing the game to the generator faster. we define the new metric {\em Nash Equilibrium Speed} to be the loss of generative model over 1000 epochs. that is proportional to the vulnerability to adversarial machine learning attack and  inversely proportional  to the resiliency of a model to adversarial machine learning attacks. 

People have noted that adding labels to the data—that is, to break it up into categories, almost always improves the performance of GANs.

Generated data, showing high features in .. which is high in all the variations. 



\input{related}
\section{Conclusion and Future Work}\label{future}

Crucially, an improvement in performance can unlock a model's usability in practice, especially in fields like detection of microarchitectural attacks, were accuracy may mean the difference between performance without security or security without performance. 
%---------End Perspectrion sections----------
%%%%%%%%% -- BIB STYLE AND FILE -- %%%%%%%%
\bibliographystyle{IEEEtran}
\bibliography{bib}
%%%%%%%%%%%%%%%%%%%%%%%%%%%%%%%%%%%%
\end{document}
