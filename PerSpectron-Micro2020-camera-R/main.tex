%
\documentclass[conference]{IEEEtran}
\IEEEoverridecommandlockouts
% The preceding line is only needed to identify funding in the first footnote. If that is unneeded, please comment it out.
%\usepackage{cite}
\usepackage{amsmath,amssymb,amsfonts}
\usepackage{algorithmic}
\usepackage{graphicx}
\usepackage{textcomp}
\usepackage{xcolor}
\def\BibTeX{{\rm B\kern-.05em{\sc i\kern-.025em b}\kern-.08em
    T\kern-.1667em\lower.7ex\hbox{E}\kern-.125emX}}
    
    
    %%%%%%%%%%%%%%%%%%%%%%%%%%%%%%%%%
\usepackage{mathptmx} % This is Times font
\usepackage{adjustbox}
\usepackage{paralist}
\usepackage{verbatim}
\usepackage{fancyhdr}
\usepackage[normalem]{ulem}
\usepackage[hyphens]{url}
\usepackage[sort,nocompress]{cite}
\usepackage[final]{microtype}
%\usepackage{flushend}
\usepackage{listings}
\usepackage{ntheorem,lipsum}
\usepackage{adjustbox}
\usepackage{amsmath}
\usepackage{amsmath}
\usepackage{longtable}
\theorembodyfont{\upshape}
\newtheorem{definition}{Definition}
\usepackage{algorithmic}
\usepackage{graphicx}
\usepackage{textcomp}
\usepackage{xcolor}
\usepackage{setspace}
\usepackage{soul,color}
\usepackage{makecell}
\usepackage{adjustbox}
\soulregister\cite7
\soulregister\ref7
\soulregister\pageref7
\usepackage{titlesec}

\usepackage{caption} 
\captionsetup[table]{skip=8pt}

\newcommand{\scheme}{\textsc{Brutus}}

\newcommand{\ignore}[1]{}
\newcommand{\nael}[1]{{\color{green} Nael: \emph{#1}}} 
\newcommand{\samira}[1]{{\color{blue} samira: \emph{#1}}} 

\newtheorem{observation}{\textbf{Observation}}
\newtheorem{remark}{\textbf{Remark}}


\newenvironment{hlbreakable}%
  {\colorbox{yellow}}%
  {}
\def\BibTeX{{\rm B\kern-.05em{\sc i\kern-.025em b}\kern-.08em
    T\kern-.1667em\lower.7ex\hbox{E}\kern-.125emX}}
%--------------------------------------------------

% %%%%%%%%%%%%%%%%%%%%%%%%%%%%%%%%%%%%
\makeatletter
\newcommand{\linebreakand}{%
  \end{@IEEEauthorhalign}
  \hfill\mbox{}\par
  \mbox{}\hfill\begin{@IEEEauthorhalign}
}
\makeatother
 \begin{document}

\title{Brutus: An Adversarial-based Perceptron Learning For Microarchitectural Attack Detection in Hardware \\

}

%\setlength{\parindent}{4em}
\setlength{\parskip}{1em}



\maketitle
%--------Perspectron sections----------------
\begin{abstract}

\end{abstract}
\begin{IEEEkeywords}
perceptron learning, microarchitectural attack defenses, secure architectures
\end{IEEEkeywords}
\section{Introduction} 
%In hardware we are limited to one layer neural network (Perceptron). Perceptron can only learn linearly separable functions.  




% showed that using carefully selected detailed microarchitectural features and a simple  single-layered perceptron can provide a readily implementable solution. 





The performance of modern processors is heavily dependent on prediction and speculative execution throughout the pipeline.  Speculative execution, however has recently shown itself to be a two edged sword with the disclosure in early 2018 of microarchitectural attacks such as Spectre and Meltdown~\cite{spectre, meltdown}. The total cost to secure each component individually against all potential microarchitectural attack variations incur increasingly high cost in real estate, performance and energy~\cite{canella2018systematic, schwarz2019zombieload}, giving us an unsatisfying choice between expensive software mitigations or turning off these performance features altogether.  Indeed, future research in performance optimization, critical to the health of the computer industry, is also at risk if every advance risks exposing new vulnerabilities.


 Several of the mitigation either impose high performance overhead
to the system and/or only mitigate a specific variant. Detection mechanisms can provide information to the system to trigger further
mitigations to avoid their overhead unless an attack is suspected. However, this presents a particularly difficult
problem for simulation-to-real-world transfer: besides 
the
targeted vulnerable components of hardware changing in each variations of these attacks, the
system must also handle {\em evasive attacks} which poses multiple challenges for designing a resilient detector, specially resiliency against {\em adversarial machine learning attacks}.  

There are three main categories of evasion techniques: (1) bandwidth reduction (2) polymorphic evasion and (3) adversarial machine learning attacks. 
Software detectors have high performance overhead due to sampling and are also prone to bandwidth evasion~\cite{Gaudiot2020, PerSpectron}. Bandwidth evasion is based on timing the completion of all attack atomic 
tasks to fit within the sampling interval. Recent studies~\cite{PerSpectron, cyclone2019} on detection of transient, MDS and cache attacks and malware~\cite{Malware2015, ensembleRaid2015,kazdagli-16,RHMD2017} addressed the software limitations by moving the detection of
attacks to hardware, allowing thousand-times higher sampling frequency (100ms vs 1$\mu$s) without incurring performance overhead. It would be very difficult for the attacker to implement an microarcitectural attack atomic phase requiring an interval under three microseconds. 

Detection in hardware does not only make evasion based on bandwidth reduction very hard, but it is also shown to make detection more robust against polymorphic evasions~\cite{PerSpectron, cyclone2019, RHMD2017}. Polymorphic attacks is when the attacker attempts to produce different binaries implementing the attack. Because in microarchitectural attacks, attacker can implement instructions that will never get it’s result committed to the registers but force the CPU to execute instructions and leak data, it is important for security solutions 
to be able to detect attacks in the speculative execution feature 
space~\cite{wampler-19, PerSpectron}. Detection to hardware also allows the predictor to efficiently use and monitor a large set of informative microarchitectural features that is not limited to the commit. Recent study~\cite{PerSpectron} shows that low-level  features exist that are invariant under transformation and evasions by typical strategies
used by malware to evade signature based 
detectors~\cite{PaulKocher,paulKocherSpectreAttacks} and are accessible from hardware. Due to the number and nature of features used in PerSpectron, adversarial machine learning attacks described in prior hardware malware detector is potentially prohibitively expensive against PerSpectron. However, previous studies have not conclusively proved that the design cannot be evaded by adversarial machine learning attacks. 

Neural network models used in current work feature deep multi-layered  networks ({\em e.g.} RNN)  are not easily amenable to hardware due to design and runtime complexity. But Perceptron learning has shown to be implementable in hardware for various  applications including branch prediction, prefetching, replacement policies, and CPUadaptation~\cite{intelISCA2019}. Recent microarchitectures from Oracle~\cite{SPARCT4}, AMD ({\em e.g.} Bobcat, Jaguar, Piledriver, Zen, etc.), and Samsung~\cite{Mongoose,M3} are documented as featuring perceptron-based branch predictors.


Hinton~\cite{Hinton1985shape} 
finds that using the entire space of possible features in the 
training set made the 
mapping from the features’ instantiation parameters to the
object’s instantiation parameters became linear allowing the use of a simpler architecture, which could efficiently 
handle more complex images.   Similarly PerSpectron~\cite{PerSpectron} found microarchitectural features that when included in training, it could map a non-linearly separable problem to linear, which was then 
separable by perceptron and is readily implementable in hardware. The result was competitive 
with a more complex deepNN that is not easily implementable in hardware.


The science of defenses for machine learning based  detectors in hardware are somewhat less well developed. Here we consider several defensive goals. 

We propose a perceptron-based models for detection of broad range of microarchitectural attacks in hardware that outperforms current  detectors against adversarial machine learning attacks. 

%In this work, we introduce  





%and the selection of features from a larger set of event


% the attacks

% We believe that PerSpectron offers an inexpensive,
% always-on solution that can provide wide coverage against
% microarchitectural attacks. We also believe that there are
% interesting possibilities when combining PerSpectron with
% other defenses, for example to raise the alarm to activate other
% defenses to avoid their overhead unless an attack is suspected.










%  Using machine learning for detection of microarchitectural attacks, e.g., transient attacks~\cite{canella2018systematic,spectre,meltdown,koruyeh2018spectre} and cache-based covert channels~\cite{GrussFlushFlush,GrussFlushFlush, FlushReload2014Yarom,PrimeProbe2015last}, is a
%  promising mitigation direction~\cite{PerSpectron,RHMD2017, gulmezoglu2019fortuneteller,cacheBasedDetection2016Chiappetta,CloudRadar2016, BlackHatFogh,payer2016hexpads,ICCAD2015Wang, Duppel2013Zhang, mushtaq:cel-01824512}. 
%  This presents a particularly difficult
% challenge for simulation-to-real-world transfer: besides 
% the
% targeted components of hardware changing in each variations of these attacks, the
% system must also handle {\em evasive attacks} which poses multiple challenges for designing a resilient detector, specially resiliency against {\em adversarial machine learning attacks}. 


%Surveys of cybersecurity professionals indicate that this is the main reason causing low confidence in AI-based defense systems~\cite{CarbonBlack}.
\section{Background and Motivation}\label{background}
\label{motiv}
\subsection{Microarchitectural Attacks}

% \subsection{Perceptron and Microarchitectural Features}

 


% Neural network models used in software detectors, feature deep multi-layered  networks ({\em e.g.} RNN) are not easily amenable to hardware due to design and runtime complexity. But Perceptron learning has shown to be implementable in hardware for various  applications including branch prediction, prefetching, replacement policies, and CPUadaptation~\cite{intelISCA2019}. Recent microarchitectures from Oracle~\cite{SPARCT4}, AMD ({\em e.g.} Bobcat, Jaguar, Piledriver, Zen, etc.), and Samsung~\cite{Mongoose,M3} are documented as featuring perceptron-based branch predictors.


% % Hinton~\cite{Hinton1985shape} 
% % finds that using the entire space of possible features in the 
% % training set made the 
% % mapping from the features’ instantiation parameters to the
% % object’s instantiation parameters became {\em linear} allowing the use of a simpler architecture, which could efficiently 
% % handle more complex images.   Similarly 
% In this work we laverage PerSpectron which is a  hardware detector for microarchitectural attacks which uses a simple one layer neural network,  allowing the use of a simpler architecture, which could efficiently 
% handle more complex footprints of attacks.
% PerSpectron~\cite{PerSpectron} found 106 microarchitectural features that when included in training, it could map a non-linearly separable problem to linear, which was then 
% separable by {\em perceptron} and is readily implementable in hardware~\footnote{An example of such feature is the effect of misses and stalls 
% in the Fetch stage. The squashed cycles in each stage, all the ROB, IQ, and 
% Register full events, undone maps in the Rename stage, and memory order 
% violation in the IEW stage propagate back to the Fetch stage. The relationship 
% between these events' Fetch is not a simple cumulative function in an out-of-order 
% processor. }. 
% PerSpectron showed that because the formation about attacks moves around the
% processor, mutually correlated features of all components of processors should be included in the feature set in order to detect new variations of attacks.However, features such as \textit{fetch.MiscStallCycle} capture the 
% relationship.
% The result was competitive 
% with a more complex deepNN that is not easily implementable in hardware.

% \subsection{Feature selection}
% Unlike image information that is simple pixels, microarhitectural features are much more complex. PerSpectron includes features that capture the relationship between multiple features. Therefore no hidden layer is needed~\footnote{}. The weights associated to these features have the potential to be updated further, similar to hidden weights in RNNs.
% No hidden layer was necessary---the 
% mapping from the features' instantiation parameters to the object's 
% instantiation parameters became linear.



\subsection{Adversarial Attacks Against Machine Learning}

Recent literature has considered two types of adversarial machine learning attacks: black-box and white-box attacks. Under the black-box attack model, the attacker does not have access to the classification model parameters; whereas in the white-box attack model, the attacker has complete access to the model architecture and parameters, including potential defense mechanisms.
%(Papernot et al., 2017; Tramer et al., 2017; ` Carlini & Wagner, 2017). 



White-box models assume that the attacker has complete knowledge of all the classifier parameters, i.e., network architecture and weights, as well as the details of any defense mechanism. Given an input x and its associated ground-truth label y, the attacker thus has access to the loss function J(x, y) used to train the network, and uses it to compute the adversarial perturbation $\delta$. Attacks can be targeted, in that they attempt to cause the perturbed attack to be misclassified as safe target class. 

As previously mentioned, black-box adversaries have no access to the classifier
or defense parameters. It is further assumed that they do not have access to a large training dataset
but can query the targeted DNN as a black-box, i.e., access labels produced by the classifier for
specific query images. The adversary trains a model, called substitute, which has a potentially
different architecture than the targeted classifier, using a very small dataset augmented by synthetic
data labeled by querying the classifier. Adversarial examples are then found by applying any
attack method on the substitute network. It was found that such examples designed to fool the substitute often end up being misclassified by the targeted classifier~\cite{szegedy2014going, papernot2017practical}. 

%In other words, black-box attacks are  transferrable from one model to the other. 

%\subsection{Handcrafting Adversarial Attack Strategies}

% The first technique is injection in which a malicious content is injected into benign process in order to avoid detection. 

% The downside of this technique is that the malicious Dynamic-Link Library (DLL) file must be stored on disk, which exposes it to detection by regular security solutions.

% To execute a malicious Dynamic-Link Library (DLL) under another process malware writes the path of a malicious DLL into a remote process’ address space. Then, to invoke the DLL’s execution, the malware creates a remote thread from the targeted process. This technique implies that the malicious DLL is stored on a disk before injecting it into the remote process. 



% To avoid storing the DLL on disk, Reflective DLL injection technique manually map the DLL’s raw binary into virtual memory, as the Windows loader would do, but without calling the Windows API’s LoadLibrary that might be detected by tools monitoring the LoadLibrary calls. It will be enough to get the correct address of the injected export function that will fully load and map remaining components of the DLL inside the target process, e.g. ReflectiveLoader().



% examples are executing DLL under another process, reflective DLL injection, hollowing the content of a benign process to include maliciois payload, 


%DLL injection is one of the simplest and most common processes injection techniques. To execute a malicious Dynamic-Link Library (DLL) under another process malware writes the path of a malicious DLL into a remote process’ address space. Then, to invoke the DLL’s execution, the malware creates a remote thread from the targeted process. This technique implies that the malicious DLL is stored on a disk before injecting it into the remote process.Steps for DLL injection:
%:Steps for DLL injection

% Locate the target process by traversing the running processes and call OpenProcess for obtaining a handle to it.
% Allocate the space for injecting the path of the malicious DLL file to the target process with a call to VirtualAllocEx with the targeted process handle.
% Write the path of the DLL into the allocated space with WriteProcessMemory.
% Retrieve the address of LoadLibrary from kernel32.dll, that given the path to DLL, loads it into memory (does not execute it though).
% Call CreateRemoteThread passing it the address of LoadLibrary causing the injected DLL file’s path to be loaded into memory and executed.
% The downside of this technique is that the malicious DLL file must be stored on disk, which exposes it to detection by regular security solutions. Nevertheless, this technique is employed by malware developers and is widespread in the wild. For example, Poison Ivy, a popular and long-standing RAT, uses DLL injection. Poison Ivy has been involved in several APT campaigns recommending itself as a tool of choice by APT groups for espionage operations.
%https://www.deepinstinct.com/2019/09/15/malware-evasion-techniques-part-1-process-injection-and-manipulation/#:~:text=Process%20injection%20and%20manipulation%20is%20a%20prominent%20method,undetected%20and%20launch%20and%20execute%20additional%20successful%20attacks.

\subsection{Defenses}

 Hardware Malware Detectors (HMDs)~\cite{RHMD2017} proposed defenses against the proliferation of malware. They stochastically switch between different detectors. These detectors can be shown to be provably more difficult to
reverse engineer based on resent results in probably approximately
correct (PAC) learnability theory. 



Defensive distillation~\cite{papernot2016distillation} trains the classifier in two rounds using a variant of the
distillation~\cite{hinton2015distilling} method. This has the desirable effect of learning a smoother network
and reducing the amplitude of gradients around input points, making it difficult for attackers to
generate adversarial examples~\cite{papernot2016distillation}. It was, however, shown that, while defensive
distillation is effective against white-box attacks, it fails to adequately protect against black-box
attacks transferred from other networks~\cite{Carlini2017}.

A popular approach to defend against adversarial machine attack is to augment the training dataset with adversarial examples~\cite{szegedy2014going, Goodfellow2015ADVexample, moosavidezfooli2016deepfool}. Adversarial examples are generated using one or more chosen attack models and added to the training
set. This often results in increased robustness when the attack model used to generate the augmented
training set is the same as that used by the attacker. It tends to make the model more robust to white-box attacks than to black-box attacks due to gradient masking~\cite{Papernot2016TowardsTS, tramer2020ensemble}.




\section{Threat Model}

% We assume an adversarial attack model which starts with the adversary attempting to reverse engineer the classifier. We assume that
% the attacker has access to a machine with a similar detector as the
% victim machine. This allows the attacker to observe the behavior
% of the classifier for given programs (whether malware or normal
% programs). With a model of the detector, the attacker can attempt to
% generate evading malware that hide themselves by changing some
% of their characteristics (feature values). 

% Such evading mechanism
%  is known as mimicry attacks~\cite{Mimicry2006,Mimicry2007}, which can be in the form of
%  no-op insertion, code obfuscation by the attackers, or calling benign
%  functions in the middle of the malicious payload~\cite{SCRAP2013HPCA}.
% We assume that the attacker that undertakes malware rewriting
% as part of a mimicry attack is interested in maintaining reasonable
% performance of the malware. If this assumption is not true, an attacker can simply run a normal program with embedded malware,
% that advances the malware program arbitrarily slow (e.g., 1 malware
% instruction every N normal instructions where N is arbitrarily large) making detection impossible. Note that this is a limitation of all
% anomaly detectors, and not only HMDs. This assumption is also
% reasonable for important segments of malware such as: malware
% that is time sensitive (e.g., that performs covert or side-channel attacks [16, 23, 37, 42]) and malware that is computationally intensive
% such as that executing on botnets being monetized under a pay-perinstall model [9] (e.g., Spam bots or Click fraud). Such malware have
% a utility to the malware writer proportional to their performance.


\section{ Overview}\label{overview}
\subsection{General Approach}
%\subsection{Game setup}

 The microarchitectural attack detection model {\em e.i.,} PerSpectron, learns the patterns of suspicious and safe activity in a program to discern an attacks footprints.
When generating handcrafted adversarial attacks, our goal was to produce a program that capture the characteristics of the training dataset {\em i.e., } of safe programs in hardware, so that the attacks it generates look indistinguishable from the training safe data in the eyes of PerSpectron. Thus our goals as an attacker can be thought of as PerSpectron's goal in reverse. That implies that there is a game setup between our model's goal and the attackers. Our design is inspired by Generative Modeling Theory specifically GANS~\cite{goodfellow2014generative}.We first discuss the theory behind GANs and then describe our design.  


\subsection{Generative Modeling Theory}
A generative model describes how a dataset is generated in
terms of a probabilistic model. By sampling from the model,
we are able to generate new data. GAN is a framework for
estimating generative models via an adversarial process, in
which it simultaneously train two models: a generative model
that captures the data distribution, and a discriminator model
that estimates the probability that a sample came from the
training data rather than the generator. The training procedure
for generator is to maximize the probability of discriminator
making a mistake~\cite{goodfellow2014generative}.


% The Generator learns through the feedback it receives from the discriminator's classification. 
% The discriminator's goal is to determine whether a particular example is coming from the training dataset or created by the Generator. 
% %Instead of recognising the pattern, the Generator learns to create them essentially from scratch; indeed, the input into the Generator is often no more than a vector of random numbers.
%  Accordingly, each time the discriminator is fooled into classifying a generated sample, the generator knows it did something well. Conversely each time the discriminator  rejects an adversarial attack, the generator receives the feedback that it needs to improve.
% The discriminator continues to improve as well. Like any classifier, it learns from how far its predictions are from the true class. 
% So, as the generator gets better at producing realistic adversarial sample, the discriminator gets better at telling adversarial samples from original data, and both models continue to improve simultaneously. 

%GAN is a type of neural network that is able to generate realistic new data from scratch. 
GANs have shown to produce impressive images of faces, bedrooms, or birds, etc.,~\cite{}, which have never been seen. Note that GAN does not sample the closest point or the average or the best fit to a dataset. They are not maximizing likelyhood of single samples, but they are minimizing the overall distance between the seen data and the generated data.  Indeed GANs should be able to produce whatever data that it is trained to generate. However, it has never been shown to be applied in microarchitecture design. We will use GANs as part of our design to train and improve the accuracy of our hardware detector.  

% We have a theoretical understanding of why the symmetric GAN training should converge to the Nash equilibrium. We noticed that our asymmetric GAN traning has a lot higher gradient and so trianing happens much more quickly at start. Although theoretically there is a chance that the training might not converge at all, we empirically show that with the right stopping criteria, the trained model performs better than symmetric GAN training. 
% But our dreadful sacrifice leads to significant improvement in accuracy.

 In simple GAN, you have no control on what category of sample input will get produced. There is no way to direct the Generator to synthesize say an attack or safe program sample. Let alone other features such as the type of covert channel such as Flush+Reload or Flush+Flush.
With this architecture we could only control the class of examples that our DNN learned to emulate by our selection of the training samples {\em i.e., safe or suspicious}. But we could not specify any of the characteristics of the microarchitectural samples it is going to generate. Another words, GAN framework could synthesize realistic looking microarchitectural footprints of programs, but it can not control what channel type or phases of attack it produces.

% GANs are capable of producing examples ranging from simple handwritten digits to photo-realistic images of human faces. However, although we could control the domain of examples our DNN learned to emulate by our selection of the training dataset. we could not specify any of the characteristics of the data samples the gan would generate. We could not control whether it would produce, say a new sample of adversarial meltdown attack. 
 
 In image recognition, this concern may seem trivial. Because if the goal is to generate number 9, you can just keep generating until you get the number you want. However, for a domain of microarchitectural attacks the domain of possible answers gets too large for such a brute-force solution to be practical. So we need a controlling mechanism.
Our solution is inspired by CGAN~\cite{cgan}. CGAN is a generative adversarial learning whose generative and discriminator are conditioned during training by using some additional information e.g., labels. The CGAN was one of the first GAN innovations that made data generation possible.  
 
 {\scheme} has the ability to decide what kind of adversarial microarchitectural attack will be generated. We can enter the descriptive features of attacks atomic tasks' samples into our DNN generator and have it output a range of samples matching the criteria. It can greatly expedite the process of adversarial attack generation~\footnote{We are sure there are many other practical applications where the ability to generate new microarchitectural sample that matches the input executive type of our choice would a game changer.}.
 
 \subsection{Perceptron and Microarchitectural Features}

 


Neural network models used in software detectors, feature deep multi-layered  networks ({\em e.g.} RNN) are not easily amenable to hardware due to design and runtime complexity. But Perceptron learning has shown to be implementable in hardware for various  applications including branch prediction, prefetching, replacement policies, and CPUadaptation~\cite{intelISCA2019}. Recent microarchitectures from Oracle~\cite{SPARCT4}, AMD ({\em e.g.} Bobcat, Jaguar, Piledriver, Zen, etc.), and Samsung~\cite{Mongoose,M3} are documented as featuring perceptron-based branch predictors.
% Hinton~\cite{Hinton1985shape} 
% finds that using the entire space of possible features in the 
% training set made the 
% mapping from the features’ instantiation parameters to the
% object’s instantiation parameters became {\em linear} allowing the use of a simpler architecture, which could efficiently 
% handle more complex images.   Similarly 

In this work we laverage PerSpectron which is a  hardware detector for microarchitectural attacks which uses a simple one layer neural network,  allowing the use of a simpler architecture, which could efficiently 
handle more complex footprints of attacks. 
PerSpectron also included the mutually correlated features from all the components of processors in the feature set. They showed that it is essential for detecting unseen variations of attacks as the information about attacks moves around the processor.


PerSpectron~\cite{PerSpectron} algorithm extracted 106 microarchitectural features that when included in training, it could map a non-linearly separable problem to linear, which was then 
separable by {\em perceptron} and is readily implementable in hardware. An example of such feature is the effect of misses and stalls 
in the Fetch stage. The squashed cycles in each stage, all the ROB, IQ, and 
Register full events, undone maps in the Rename stage, and memory order 
violation in the IEW stage propagate back to the Fetch stage. The relationship 
between these events' Fetch is not a simple cumulative function in an out-of-order 
processor. 
However, features such as \textit{fetch.MiscStallCycle} capture the 
relationship.
The result was competitive 
with a more complex deepNN that is not easily implementable in hardware. Thus, 

\begin{note}

A Perceptron using invariant microarchitectural features of the pipeline can be set up to play an adversarial game with a much more complex deepNN. 

% 
\end{note}
 
In the next section, we explain \scheme and propose an asymmetric GAN training Perceptron automatically for adversarial examples of microarchitectural attacks. We then explain how to control the type of adversarial microarchitectural attack sample generated, followed by a method using the knowledge from the correlations between pipeline features to adaptively train the Perceptron for adversarial examples and reduces the transferability of the Perceptron's parameters. 

\begin{figure*}
\centering
\includegraphics[width=0.98\textwidth, height=0.4\textheight]{PerSpectron-Micro2020-camera-R/img/blockDiagram.png}

\caption{\scheme{}  }

\label{fig:algdiagram}
\end{figure*}

\subsection{{\scheme} Architecture}

{\scheme} uses a deep neural network (DNN) as generator and PerSpectron,  as the discriminator. This differs with original  GANs~\cite{goodfellow2014generative} which uses same convolutional neural
networks (CNNs) in both the discriminator and the generator
architecture of GAN.  The DNN and PerSpectron play an
adversarial game. We introduce this architecture as an {\em asymmetric} GAN, in which a one layer neural network with large number of detailed microarchitectural features, tries to determine whether a particular example is coming from the training dataset or created by the DNN. 

The DNN learns through the feedback it receives from the PerSpectron's classification. 
 Accordingly, each time the PerSpectron is fooled into classifying a generated sample, the DNN knows it did something well. Instead of recognising the pattern, the DNN learns to create them essentially from scratch; indeed, the input into the DNN is mainly a vector of random numbers and labels which we describe in next section.  Conversely each time the PerSpectron  rejects an adversarial attack, the DNN receives the feedback that it needs to improve.

The PerSpectron continues to improve as well. Like any classifier, it learns from how far its predictions are from the true class. 
So, as the DNN generator gets better at producing realistic adversarial samples, the PerSpectron gets better at telling generated samples from original data, and both models continue to improve simultaneously. 


%The equation of loss function is given by [min max equation]. 




% The Generator learns through the feedback it receives from
% the discriminator’s classification. The discriminator’s goal is
% to determine whether a particular example is coming from the
% training dataset or created by the Generator. Accordingly, each
% time the discriminator is fooled into classifying a generated
% sample, the generator knows it did something well. Conversely
% each time the discriminator rejects an adversarial attack, the
% generator receives the feedback that it needs to improve. The
% discriminator continues to improve as well. Like any classifier,
% it learns from how far its predictions are from the true class. So,
% as the generator gets better at producing realistic adversarial
% sample, the discriminator gets better at telling adversarial
% samples from original data, and both models continue to
% improve simultaneously.
%A conditional GAN allows us to direct the generator to synthesize the adversarial example we want.

 




% \subsection{Data Transmit And Recovery Channel Control}
\subsection{Adversarial Attack Type Control}~\label{TypeControl}
 Our DNN generator learns to produce realistic microarchitectural footprint examples for each label in the training dataset and the PerSpectron learns to distinguish generated sample-label pair from real label-sample pairs.   
 In contrast to a design where the Disriminator learns to assign a correct label to each real example in addition to distinguishing real example from fake, our PerSpectron discriminator does not learn to identify which class is which.  It learns to accept matching pair from seen database while rejecting pairs that are mistmatched and pairs in which there are generated by DNN. 
 
 The DNN uses the noise vector and label to synthesize an adversarial example given that or conditioned on the value of the label. The goal of this fake example is to look in the eyes of Discriminator as close as possible to a seen {\em i.e., attack} example for the given label.  
 Accordingly in order to fool the PerSpectron it is not enough for our DNN generator to produce realistic looking footprints of attacks and safe programs. The sample it generates also need to match their labels. After our DNN generator is fully trained, this then allows us to specify what attack type we want to synthesize by passing it the desired label. 
 
 The PerSpectron receives seen samples from dataset  with labels, and generated examples from DNN with the label used to generate them.  PerSpectron learns two things: 
 On the seen data-label pairs it learn
 how to recognize the seen attack and safe data from the generated ones and, second, how to recognize matching pairs generated by the DNN. On the generator-based examples, PerSpectron learns to recognize generated program-labeled pairs, thereby learning to tell them apart from the seen ones. The PerSpectron outputs a single probability indicating its conviction that input is a seen attack, matching pair. 
 
\begin{note}
{In contrary to the original PerSpectron paper, the PerSpectron's goal here is to learn to reject all generated examples and all examples that fail to match their label, while accepting all seen example-label pairs of microarchitectural attacks.}
 \end{note}
 
 Note that for each generated example the same label is passed to both the DNN and the PerSpectron. Also note that the DNN is never explicitly trained to reject mismatched pairs by being trained on the seen examples with mismatching labels; its ability to identify mismatched pairs is a by-product of being trained to accept only seen matching pairs. 
 
 
\noindent\textbf{Train the DNN}:
For each training iteration:

%\begin{compactitem}
\begin{itemize}  [topsep=0pt,parsep=0pt,partopsep=0pt, label={--}, leftmargin=*] %[leftmargin=*] noitemsep
\item  The DNN takes a new random noise z and three labels: A the transmission channel type of the attack {\em i.e., meltdown-type}, A the recovery channel type of the attack {\em i.e., prime+probe-type} and the class target $t$ {\em i.e., safe or suspicious}. DNN generates an example $x^{\star}$ that strives to be both an adversarial attack (or a realistic safe program) and a convincing match for the labels $F$ and $R$.  

\item PerSpectron network classifies $x^{\star}$ into 0/1. One for matched seen sample-labels from the dataset and 0 for unmatched sample-labels and generated sample-labels.    

% \item Compute the classification error and backpropogate the total error to update PerSpectron's  weights, seeking to minimize it's classification error. 

\item  The classification error of PerSpectron will be computed and backpropogated to update the DNN weights, seeking to maximize the PerSpectron's error. 

\end{itemize}

%\end{compactitem}

\noindent\textbf{Train the PerSpectron:}
For each iteration:

%\begin{compactitem}
\begin{itemize} [topsep=0pt,parsep=0pt,partopsep=0pt, label={--}, leftmargin=*]
%[labelindent=0pt]%[leftmargin=*] noitemsep
\item We choose a seen program sample $x$ from the training dataset with its two labels: the transmission channel type $F$ of the attack {\em i.e., meltdown-type}, A the recovery channel type $R$ of the attack {\em i.e., prime+probe-type} and the class target $t$ {\em i.e., safe or suspicious}. 
 
\item DNN gets a new random noise vector $z$ and label $F$ and $R$ and  synthesize an adversarial example  $x^{\star}$.
 
\item PerSpectron receives the seen sample example $x$ with label $F$ and $R$ and  the adversarial generated sample-label $x^{\star}$ and the labels that were used to generate them $F$ and $R$. For both examples PerSpectron outputs a probability  indicating whether the input example was a seen data from dataset, matching its label pair and  for a generated data or unmatched pairs (0/1). 
 
\item The classification error of PerSpectron will be computed and backpropogated  to update it's weights, seeking to minimize it's classification error. 
 
\end{itemize}
%\end{compactitem}

% 
The training ends when our GANs reaching Nash equilibrium. Which happens when the following conditions are met: DNN produces fake examples that are not distinguishable from the seen attacks in the training dataset. The PerSpectron can at best randomly  randomly guess whether a particular example is a seen attack or a generated adversarial one. All examples produced by our DNN at this points with feeding various categories of attacks to it, are adversarial attack samples which carry the footprints of microarchitectural attacks but are very hard to distinguish. 


In the next section we explain how to use the generated adversarial samples to train the original PerSpectron classifier. 

\subsection{Training on Adversarial Examples}

We want to use GANs to create a large dataset, but we need a large dataset to train the GAN in the first place. GANs are notoriously hard to  train. As with any other cutting-edge field, opinions about what is the best approach are evolving specially no study has been done on microarchitectural data.
On the other hand, attack programs include safe code segments. This is like feeding a GAN a partial frame of an image which is common between different faces. Eventually GANs learn to identify the exact footprints of certain class and will generate adversarial based on those but the safe code segments of attack programs slows down the training of GANs leading to further need for more samples, a catch 22 situation. We solve this problem through the following:


First, we use standard data augmentation using program synthesising to create a larger dataset of all the attack, this samples does not add to the diversity of examples as we discussed in section~\ref{intro} and ~\ref{results} but it helps our GAN to converge faster. 

Second, instead of training our GAN on the whole attack program, we train on samples taken from attacks atomic tasks only.  ”Atomic
Tasks” are operations that, if interrupted, will disable the attack: For example, Flushing cache lines, mistraining branch predictor  or
Attempting to infer the secret byte that is loaded into cache.
This enables the GAN to automatically generate adversarial examples of tasks related to the attacks regardless of the safe code segments presented in those attacks.

Third, we use this dataset to train a GAN to create adversarial attack sample-pairs as we explained in section~\ref{TypeControl}. We input two labels to our DNN, the transmission type of the attack {\em i.e., meltdown-type } and the recover type of the attack {\em i.e., prime+probe-type}. The generator will generate label-conditioned adversarial attacks that misleads the PerSpectron classifier. 
Finally, we use the augmented dataset from step 1 along with the GAN-produced adversarial samples from step 2 to train the PerSpectron. 

 

  

 
\subsection{Adaptive Adversarial Learning}

We now that many microarchitectural features are mutually correlated in different components of processors~\cite{PerSpectron}. This is due to the nature of components of pipeline, and their interleaved functionality in an out of order superscaler process. For example, \textit{IcacheSquashes}, \textit{MiscStallCycles} and \textit{PendingTrapStallCycles} 
are mutually decorrelated in the fetch unit but they have high correlation with stalls 
and traps in other components, {\em i.e.} \textit{commit.NonSpecStalls}, 
\textit{lsq.thread0.rescheduledLoads} and \textit{dcache.blocked:no\_mshrs}. This property has been used in prior work~\cite{PerSpectron} to produce {\em replicated detectors} to improve the detection as the attack footprint moves in the pipeline.


Many considered random feature selection
strategy and showed that it can improve the security and robustness of forensic detectors and
standard ML-based to mitigate the adversarial machine learning attacks ~\cite{nowroozi2020survey, secureDetection2019}.
 In this work we use removal of mutually correlated features of different components of processor to reduce the transferability of classification parameters, and increase the security of our 
detection. 

We apply adaptive feature weight elimination during adversarial training, in country to previous works which dynamically switched between different classifier. Our adaptation is embedded in Perceptrons weight from the training phase.  

Our weight clipping is very simple. We first group the microarchitectural features into mutually correlated sets {\em e.i.,} one feature per component of pipeline and uncore units. During training we add a layer between generator and discriminator. This layer chooses one feature from each group and sets the corresponding weight output values from the generator to zero.

Incredibly, this simple addition drastically improves accuracy on adversarial attacks, by ensuring that network doesn't become overdependant on certain units or groups of features that in effect just remember observation from training set. With weight clipping perceptron can not rely too much on any one units or features of the processor. Therefore, perceptron weight updates are more evenly distributed throughout the adversarial training. This makes perceptron learning much better at generalizing to unseen attacks, because the PerSpectron has been trained to produce accurate predictions even under unfamiliar conditions, such as those caused by dropping a highly correlated feature. At test time the weight clipping layer doesn't remove any weight so that the full Perceptron is used for prediction. 





\subsection{\scheme{} Equilibrium State }
In original GAN, because the two agents only competing against each other, it makes sense that the Generator's loss would be a negative of the discriminator. In our case a perceptron discriminator is competing against a DNN as the generator.  So far we are never given a clear set of conditions under which the training has finished in practice.
We use a stopping criteria that significantly improve on the loss function, which are now interpret-able and provide clearer stopping criteria. We try to minimize the distance between the expectation of the real attack and the expectation of the generated distribution.  

As mentioned we limit the sampling of our microarchitectural attacks to their {\em atomic tasks} the intervals in which the essential operations of attacks are being execute such as training the branch predictor, prime and probing the cache, etc.  

\begin{figure}[ht!] 
\centering
\includegraphics[width=0.45\textwidth]{PerSpectron-Micro2020-camera-R/img/saturation.png}
\vspace*{-4mm}
\caption{ The y-axis is the loss function for the DNN, whereas the horizontal axis is the PerSpectron outcome for the likelihood of the generated example.  
 % \scheme{} 
}
\label{fig:sature}
\end{figure}
\subsection{Hardware Design}
Regardless of the order
and the position of each feature, the k-sparse representation
are distinctive, and only contain 0/1 values
% {\small
% \begin{align*}
% \vspace{0.05in}
% \tiny{[f1 = ReadResp, 
%  f2 = commitNonSpecStalls, 
%  f3 = PendingQuiesceStallCycles, 
%  f4 = CleanEvict]}
% \end{align*}}
% {\small
% \begin{align*}
% suspicious: <0,1,0,0> \\
% suspicious: <1,0,1,0> \\
% suspicious: <0,0,0,1>  \\ 
% safe: <1,1,0,0> 
% \end{align*}
% }
 Computing the perceptron output. Multiplication is not necessary to compute the dot product. Simply add the weight when the input bit is 1 and subtract when bit is -1. Only the sign bit of the result is needed to make a prediction. Prediction happens in hardware and in parallel to the execution.
No additional overhead for hardware counters. Hundreds of counters are included (for debug and verification). Simpler than perceptron-based branch predictors used in mobile CPUs (see e.g. the Samsung Exynos paper in ISCA2020).

\textcolor{red}{DMT: This is definitely a useful data point, but not compelling.  It's not clear 
whether the reader will place equal value on an attack predictor and a branch predictor, and be willing
to assign them equal space.  Thus, may need more specifics about area/power/transistors.}
\section{Methodology}\label{method}
We use the gem5~\cite{gem5} cycle-level simulator.
Table~\ref{table:GEM5} gives the parameters of the simulated architecture. 
We used the FANN C library~\cite{FANNnissen03} to implement PerSpectron to 
work with gem5. We used the scikit-learn python library to measure the performance 
of other machine learning models, including $K$-nearest neighbors, Decision Tree, 
Logistic Regression, and Neural Network. 


\noindent\textbf{Data:}  
For speculative attacks, we simulate SpectreV1,SpectreV2, SpectreRSB, 
Meltdown, breakingKSLR (based on Meltdown) and CacheOut attack. For cache 
attacks, we run Flush+Flush, Flush+Reload, and Prime+Probe along with their 
respective profiling-calibration phases. Calibration programs threshold distinguishing 
cache hits and cache misses based on different cache attack techniques~\cite{PrimeProbe2015last,FlushReload2014Yarom, GrussFlushFlush}~\footnote{We limit the sampling of our microarchitectural attacks to their {\em atomic tasks} the intervals in which the essential operations of attacks are being executed such as training the branch predictor, filling the reorder buffer with dependent instructions, prime and probing the cache, etc.  
}
 
 
For benign programs, we run individual SPEC CPU 2006 applications~\cite{spec2006}. 
The workloads include C compression programs modified to do most work in memory 
(rather than I/O), optimization scheduling, Ethernet network simulator, high-rank 
artificial intelligence programs, discrete event simulation, gene sequence protein 
analysis, the A* algorithm, and more. 
 
We have written an interface to gem5 that dumps statistics once every 10K, 50K, and 100K 
instructions, and samples all event counters for each program. Using this tool, we 
collect multi-dimensional time-series traces of applications.

\noindent \textbf{Statistics:}
We examined 1159 microarchitectural counters, including features related to 
cycle accounting and micro-operation ($\mu$op) flow, stall decomposition, 
precise memory access events, latency events, precise branch events, core memory 
access events, and other core and uncore events. 
For each event (if applicable) we measure total number, cycles, rate, average and distribution. 
For each counter, we maintain a maximum possible value for each sampling simulation point. 
Each statistic is normalized over the maximum value of that counter.

We include events related to stalls and differentiate between types of stalls and 
among various points in the pipeline. We differentiate between read and write 
latency, and what came from the master ports and was forwarded to the slave 
(requests and responses). Likewise, what came from the slave and was forwarded 
to the master. We also examined features related to energy consumption in different 
microarchitectural units and pipeline stages. Also tracked are complete power state 
machine statistics such as Active, Idle, Active Power-Down, Precharge Power-Down, 
and Self-Refresh values. 

\noindent \textbf{Training and Validation:} We train the perceptrons for 1000 epochs, or 
until the training error falls below 0.4.  
We used 3-fold stratified splitting with randomization. The class split for each fold for 100K setting is as follows:  \\
Class: [benign malicious]\\
train - [4820, 1500]  | test - [2420, 740]

% In all experiments and evaluations we used 3-fold cross validation with case separation at
% the patient level and each fold contained a balanced number
% of attacks, and safe programs.

% % Since
% % our dataset was too small for effective training, we incorporated classic augmentation for the training process. We employed the DCGAN architecture to train each  class separately, using the same 3-fold cross validation process and the
% % same data partition. 

The GAN model we used is a variation of Deep Convolutions GAN. We chose class conditioned GAN~\cite{} as our main frame GAN network. We had to adjust the dimensions of hidden layers and the dimensions of the output from the Generator and input into the Discriminator PerSpectron. Then we used the trained PerSpectron to classify unseen test set.  
  
% Training dataset-- The dataset of all safe programs and real attacks that we want the generator to learn to emulate with near-perfect quality. This dataset serves as input($x$) to the generator network. 

% Random noise vector-- The raw input ($z$) to the Generator network. This input is a vector of random numbers that the generator uses as a starting point for generating adversarial examples. 

% Discriminator network-- The discriminator takes as input either the real examples ($x$) coming from the training set or an adversarial example $x^{\star}$ produced by the generator. For each example, the discriminator determines and outputs the probability of whether the example is adversarial.

% Iterative training/tuning--For each of the discriminators predictions, we determine how good it is--much as we would for a regular classifier-- and use results to attractively tune the discriminator and the generator networks through backpropogation; the discriminator's weights and biases are updated to maximize its classification and accuracy ( maximizing the probability of correct prediction: ($x$) as attack and $x^{\star}$ is safe. 
% The generator's weights and biases are updated to maximize the probability that the discriminator missclassifies $x^{\star}$ as safe program. 



\begin{table}[!htbp]
\small
\centering
\begin{tabular}{|c|}
\hline
\textbf{Architecture}  \\ \hline
X86 O3CPU 1 core Single Thread at 2.0GHz \\ \hline

\textbf{Core}  \\ \hline
Tournament branch predictor\\
16 RAS entries, 4096 BTB entries\\
LQEntries=32, SQEntries=32, ROBEntries=192\\
fetch/dispatch/issue/commit width=8\\
numPhysIntRegs=256,numPhysFloatRegs=256 \\ \hline

\textbf{L1 I-Cache}  \\ \hline
32KB, 64B line, 4-way \\ \hline

\textbf{L1 D-Cache}  \\ \hline
64KB, 64B line, 8-way \\ \hline

\textbf{Shared L2 cache}  \\ \hline
2MB bank, 64B line, 8-way,  \\
mshrs=20, tgtsPerMshr=12, writeBuffers=8  \\ 
tagLatency=20, dataLatency=20, responseLatency=20\\ \hline
\end{tabular}
\caption{Parameters of simulated architecture}
  \label{table:GEM5}
\end{table}
\section{Security Analysis}\label{sec:whac}\label{sec:varSpec}

\subsection{Developing Evasive Microarchitectural Attacks }
\section{Experimental Evaluation}\label{results}

% No attack was able to leak, but number of false positive increased. 



Accuracy on 3-fold cross validation improved by 7\%


We take an attack sample first and project it back to the latent space $z$.
We use the generator $G$ to generate a similar example to $x$, called ${x^{\star}}$ by $G(z)$. Then we use the classifier $C$ to classify the example $C(x^{\star})$, which generally already tends to misclassify way less than running the classification directly on $x$. 

Performance of the new classifier does drop on the training set, but accuracy improves on test set including adversarial attacks. Figure shows APL achieves significant improvement in classification accuracy as the number of training examples increases, especially synthetic examples produced by DCGAN. The dotted line depicts classification performance for classic data augmentation using program synthesizers. The performance improves as the quantity of new (augmented) training examples increases; however, the improvement plateaus around the accuracy of 80\%, beyond which additional examples fail to yield improvement. 

The dashed line shows the additional increase in accuracy achieved by augmenting the dataset using GAN-produced synthetic examples. Starting from the point beyond which additional classically augmented examples stopped improving accuracy, we added syntetic data generated by our GAN. The classification performance improved from 83\% to pver 90\%, demonstrating the usefulness of \scheme. 

Figure shows the loss value of generator vs training epoch for a range of features. Using a few high level features would result in discriminator losing the game to the generator faster. we define the new metric {\em Nash Equilibrium Speed} to be the loss of generative model over 1000 epochs. that is proportional to the vulnerability to adversarial machine learning attack and  inversely proportional  to the resiliency of a model to adversarial machine learning attacks. 

People have noted that adding labels to the data—that is, to break it up into categories, almost always improves the performance of GANs.

Generated data, showing high features in .. which is high in all the variations. 


Table shows the attack success rate under attacks, inclduing {\scheme} and other synthetization base technique and original PerSpectron. We can see that {\scheme} outperforms others. 
Again we have to emphesize that unlike other perturbation techniques, {\scheme} cannot limit the size of perturbation by nature so it is not surprizing higher 
\section{Related Work}\label{related}

\section{Conclusion and Future Work}\label{future}

Crucially, an improvement in performance can unlock a model's usability in practice, especially in fields like detection of microarchitectural attacks, were accuracy may mean the difference between performance without security or security without performance. 
%---------End Perspectrion sections----------
%%%%%%%%% -- BIB STYLE AND FILE -- %%%%%%%%
\bibliographystyle{IEEEtran}
\bibliography{bib}
%%%%%%%%%%%%%%%%%%%%%%%%%%%%%%%%%%%%
\end{document}
