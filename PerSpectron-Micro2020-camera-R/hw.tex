\subsection{Hardware Design}
Regardless of the order
and the position of each feature, the k-sparse representation
are distinctive, and only contain 0/1 values
% {\small
% \begin{align*}
% \vspace{0.05in}
% \tiny{[f1 = ReadResp, 
%  f2 = commitNonSpecStalls, 
%  f3 = PendingQuiesceStallCycles, 
%  f4 = CleanEvict]}
% \end{align*}}
% {\small
% \begin{align*}
% suspicious: <0,1,0,0> \\
% suspicious: <1,0,1,0> \\
% suspicious: <0,0,0,1>  \\ 
% safe: <1,1,0,0> 
% \end{align*}
% }
 Computing the perceptron output. Multiplication is not necessary to compute the dot product. Simply add the weight when the input bit is 1 and subtract when bit is -1. Only the sign bit of the result is needed to make a prediction. Prediction happens in hardware and in parallel to the execution.
No additional overhead for hardware counters. Hundreds of counters are included (for debug and verification). Simpler than perceptron-based branch predictors used in mobile CPUs (see e.g. the Samsung Exynos paper in ISCA2020).
